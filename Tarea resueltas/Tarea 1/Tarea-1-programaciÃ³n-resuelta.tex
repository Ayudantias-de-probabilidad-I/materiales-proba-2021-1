% Options for packages loaded elsewhere
\PassOptionsToPackage{unicode}{hyperref}
\PassOptionsToPackage{hyphens}{url}
%
\documentclass[
]{article}
\usepackage{lmodern}
\usepackage{amssymb,amsmath}
\usepackage{ifxetex,ifluatex}
\ifnum 0\ifxetex 1\fi\ifluatex 1\fi=0 % if pdftex
  \usepackage[T1]{fontenc}
  \usepackage[utf8]{inputenc}
  \usepackage{textcomp} % provide euro and other symbols
\else % if luatex or xetex
  \usepackage{unicode-math}
  \defaultfontfeatures{Scale=MatchLowercase}
  \defaultfontfeatures[\rmfamily]{Ligatures=TeX,Scale=1}
\fi
% Use upquote if available, for straight quotes in verbatim environments
\IfFileExists{upquote.sty}{\usepackage{upquote}}{}
\IfFileExists{microtype.sty}{% use microtype if available
  \usepackage[]{microtype}
  \UseMicrotypeSet[protrusion]{basicmath} % disable protrusion for tt fonts
}{}
\makeatletter
\@ifundefined{KOMAClassName}{% if non-KOMA class
  \IfFileExists{parskip.sty}{%
    \usepackage{parskip}
  }{% else
    \setlength{\parindent}{0pt}
    \setlength{\parskip}{6pt plus 2pt minus 1pt}}
}{% if KOMA class
  \KOMAoptions{parskip=half}}
\makeatother
\usepackage{xcolor}
\IfFileExists{xurl.sty}{\usepackage{xurl}}{} % add URL line breaks if available
\IfFileExists{bookmark.sty}{\usepackage{bookmark}}{\usepackage{hyperref}}
\hypersetup{
  pdftitle={Tarea 1- problema de programación},
  pdfauthor={Zyanya Tanahara},
  hidelinks,
  pdfcreator={LaTeX via pandoc}}
\urlstyle{same} % disable monospaced font for URLs
\usepackage[margin=1in]{geometry}
\usepackage{color}
\usepackage{fancyvrb}
\newcommand{\VerbBar}{|}
\newcommand{\VERB}{\Verb[commandchars=\\\{\}]}
\DefineVerbatimEnvironment{Highlighting}{Verbatim}{commandchars=\\\{\}}
% Add ',fontsize=\small' for more characters per line
\usepackage{framed}
\definecolor{shadecolor}{RGB}{248,248,248}
\newenvironment{Shaded}{\begin{snugshade}}{\end{snugshade}}
\newcommand{\AlertTok}[1]{\textcolor[rgb]{0.94,0.16,0.16}{#1}}
\newcommand{\AnnotationTok}[1]{\textcolor[rgb]{0.56,0.35,0.01}{\textbf{\textit{#1}}}}
\newcommand{\AttributeTok}[1]{\textcolor[rgb]{0.77,0.63,0.00}{#1}}
\newcommand{\BaseNTok}[1]{\textcolor[rgb]{0.00,0.00,0.81}{#1}}
\newcommand{\BuiltInTok}[1]{#1}
\newcommand{\CharTok}[1]{\textcolor[rgb]{0.31,0.60,0.02}{#1}}
\newcommand{\CommentTok}[1]{\textcolor[rgb]{0.56,0.35,0.01}{\textit{#1}}}
\newcommand{\CommentVarTok}[1]{\textcolor[rgb]{0.56,0.35,0.01}{\textbf{\textit{#1}}}}
\newcommand{\ConstantTok}[1]{\textcolor[rgb]{0.00,0.00,0.00}{#1}}
\newcommand{\ControlFlowTok}[1]{\textcolor[rgb]{0.13,0.29,0.53}{\textbf{#1}}}
\newcommand{\DataTypeTok}[1]{\textcolor[rgb]{0.13,0.29,0.53}{#1}}
\newcommand{\DecValTok}[1]{\textcolor[rgb]{0.00,0.00,0.81}{#1}}
\newcommand{\DocumentationTok}[1]{\textcolor[rgb]{0.56,0.35,0.01}{\textbf{\textit{#1}}}}
\newcommand{\ErrorTok}[1]{\textcolor[rgb]{0.64,0.00,0.00}{\textbf{#1}}}
\newcommand{\ExtensionTok}[1]{#1}
\newcommand{\FloatTok}[1]{\textcolor[rgb]{0.00,0.00,0.81}{#1}}
\newcommand{\FunctionTok}[1]{\textcolor[rgb]{0.00,0.00,0.00}{#1}}
\newcommand{\ImportTok}[1]{#1}
\newcommand{\InformationTok}[1]{\textcolor[rgb]{0.56,0.35,0.01}{\textbf{\textit{#1}}}}
\newcommand{\KeywordTok}[1]{\textcolor[rgb]{0.13,0.29,0.53}{\textbf{#1}}}
\newcommand{\NormalTok}[1]{#1}
\newcommand{\OperatorTok}[1]{\textcolor[rgb]{0.81,0.36,0.00}{\textbf{#1}}}
\newcommand{\OtherTok}[1]{\textcolor[rgb]{0.56,0.35,0.01}{#1}}
\newcommand{\PreprocessorTok}[1]{\textcolor[rgb]{0.56,0.35,0.01}{\textit{#1}}}
\newcommand{\RegionMarkerTok}[1]{#1}
\newcommand{\SpecialCharTok}[1]{\textcolor[rgb]{0.00,0.00,0.00}{#1}}
\newcommand{\SpecialStringTok}[1]{\textcolor[rgb]{0.31,0.60,0.02}{#1}}
\newcommand{\StringTok}[1]{\textcolor[rgb]{0.31,0.60,0.02}{#1}}
\newcommand{\VariableTok}[1]{\textcolor[rgb]{0.00,0.00,0.00}{#1}}
\newcommand{\VerbatimStringTok}[1]{\textcolor[rgb]{0.31,0.60,0.02}{#1}}
\newcommand{\WarningTok}[1]{\textcolor[rgb]{0.56,0.35,0.01}{\textbf{\textit{#1}}}}
\usepackage{graphicx,grffile}
\makeatletter
\def\maxwidth{\ifdim\Gin@nat@width>\linewidth\linewidth\else\Gin@nat@width\fi}
\def\maxheight{\ifdim\Gin@nat@height>\textheight\textheight\else\Gin@nat@height\fi}
\makeatother
% Scale images if necessary, so that they will not overflow the page
% margins by default, and it is still possible to overwrite the defaults
% using explicit options in \includegraphics[width, height, ...]{}
\setkeys{Gin}{width=\maxwidth,height=\maxheight,keepaspectratio}
% Set default figure placement to htbp
\makeatletter
\def\fps@figure{htbp}
\makeatother
\setlength{\emergencystretch}{3em} % prevent overfull lines
\providecommand{\tightlist}{%
  \setlength{\itemsep}{0pt}\setlength{\parskip}{0pt}}
\setcounter{secnumdepth}{-\maxdimen} % remove section numbering

\title{Tarea 1- problema de programación}
\author{Zyanya Tanahara}
\date{9/20/2020}

\begin{document}
\maketitle

\hypertarget{crea-un-vector-de-15-nuxfameros-los-uxfaltimos-5-deben-ser-la-sucesiuxf3n-del-1-al-5.-impruxedmelo.}{%
\subsection{1. Crea un vector de 15 números, los últimos 5 deben ser la
sucesión del 1 al 5.
Imprímelo.}\label{crea-un-vector-de-15-nuxfameros-los-uxfaltimos-5-deben-ser-la-sucesiuxf3n-del-1-al-5.-impruxedmelo.}}

\begin{Shaded}
\begin{Highlighting}[]
\NormalTok{vect <-}\StringTok{ }\KeywordTok{c}\NormalTok{(}\DecValTok{1}\OperatorTok{:}\DecValTok{10}\NormalTok{,}\DecValTok{1}\OperatorTok{:}\DecValTok{5}\NormalTok{)}
\NormalTok{vect}
\end{Highlighting}
\end{Shaded}

\begin{verbatim}
##  [1]  1  2  3  4  5  6  7  8  9 10  1  2  3  4  5
\end{verbatim}

\hypertarget{toma-el-vector-que-creaste-y-construye-con-uxe9l-una-matriz-con-3-renglones.-impruxedmela.}{%
\subsection{2. Toma el vector que creaste y construye con él una matriz
con 3 renglones.
Imprímela.}\label{toma-el-vector-que-creaste-y-construye-con-uxe9l-una-matriz-con-3-renglones.-impruxedmela.}}

\begin{Shaded}
\begin{Highlighting}[]
\NormalTok{matriz1 <-}\StringTok{ }\KeywordTok{matrix}\NormalTok{(vect, }\DataTypeTok{nrow =} \DecValTok{3}\NormalTok{)}
\NormalTok{matriz1}
\end{Highlighting}
\end{Shaded}

\begin{verbatim}
##      [,1] [,2] [,3] [,4] [,5]
## [1,]    1    4    7   10    3
## [2,]    2    5    8    1    4
## [3,]    3    6    9    2    5
\end{verbatim}

\hypertarget{toma-el-vector-que-creaste-y-construye-con-uxe9l-un-data-frame-con-5-renglones-que-tenga-un-nombre-en-cada-columna-y-cuya-uxfaltima-columna-sea-la-serie-del-1-al-5.}{%
\subsection{3. Toma el vector que creaste y construye con él un data
frame con 5 renglones, que tenga un nombre en cada columna y cuya última
columna sea la serie del 1 al
5.}\label{toma-el-vector-que-creaste-y-construye-con-uxe9l-un-data-frame-con-5-renglones-que-tenga-un-nombre-en-cada-columna-y-cuya-uxfaltima-columna-sea-la-serie-del-1-al-5.}}

\begin{Shaded}
\begin{Highlighting}[]
\NormalTok{matriz2 <-}\StringTok{ }\KeywordTok{matrix}\NormalTok{(vect, }\DataTypeTok{nrow =} \DecValTok{5}\NormalTok{)}
\NormalTok{data <-}\StringTok{ }\KeywordTok{data.frame}\NormalTok{(}\StringTok{"Nombre 1"}\NormalTok{ =}\StringTok{ }\NormalTok{matriz2[,}\DecValTok{1}\NormalTok{], }\StringTok{"Nombre 2"}\NormalTok{ =}\StringTok{ }\NormalTok{matriz2[,}\DecValTok{2}\NormalTok{],}
                   \StringTok{"Nombre 3"}\NormalTok{ =}\StringTok{ }\NormalTok{matriz2[,}\DecValTok{3}\NormalTok{])}
\end{Highlighting}
\end{Shaded}

\hypertarget{toma-al-vector-original-construye-otro-vector-que-tenga-los-elementos-mayores-a-4-de-uxe9ste-e-impruxedmelo.-debe-devolverlos-en-el-orden-del-vector-original.}{%
\subsection{4. Toma al vector original, construye otro vector que tenga
los elementos mayores a 4 de éste e imprímelo. Debe devolverlos en el
orden del vector
original.}\label{toma-al-vector-original-construye-otro-vector-que-tenga-los-elementos-mayores-a-4-de-uxe9ste-e-impruxedmelo.-debe-devolverlos-en-el-orden-del-vector-original.}}

\begin{Shaded}
\begin{Highlighting}[]
\NormalTok{m4 <-}\StringTok{ }\NormalTok{vect }\OperatorTok{>}\DecValTok{4}
\NormalTok{mayora4 <-}\StringTok{ }\NormalTok{vect[m4]}
\NormalTok{mayora4}
\end{Highlighting}
\end{Shaded}

\begin{verbatim}
## [1]  5  6  7  8  9 10  5
\end{verbatim}

\end{document}
